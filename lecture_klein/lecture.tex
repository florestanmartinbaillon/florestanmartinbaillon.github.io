documentclass{article}
% \usepackage{mypackage}
\usepackage{hyperref}
\begin{document}

% Macro
\newcommand\C{\mathbb{C}}
\newcommand\R{\mathbb{R}}
\newcommand\N{\mathbb{N}}
\newcommand\D{\mathbb{D}}
\newcommand\Z{\mathbb{Z}}
\newcommand\PP{\mathbb{P}}

\section{Guide de lecture}
\label{sec:guide_de_lecture}

\subsection{Semaine 1}
\label{sub:semaine_1}

Je vous propose de lire la partie 2 du texte de Perrin
et conjointement la partie 1 du texte de Lê.
Pour les notions nécessaires de géométrie algébrique
des courbes planes, je vous renvois 
au 2 premiers chapitres du livre de Fischer.
Il n'est bien sûr pas nécessaire de tout comprendre
(en particulier les preuves en première lecture),
nous parlerons ensemble de ce que vous n'avez pas compris.

Et sinon, pour se convaincre de la richesse de cet objet,
je me dois de mentionner le livre édité par Silvio Levy,
dont chaque chapitre présente un aspect différent de la quartique
de Klein.
Les maths sont bien sûr beaucoup trop dur pour vous, mais peut-être
que les introductions des chapitres vous donneront envie d'en savoir plus un
jour !
Il contient aussi une traduction de l'article original de Klein qui "découvre"
sa quartique.

(cf Bibliographie \ref{sub:bibliographie}).

\subsection{Semaine 2}
\label{sub:semaine_2}

Comprendre la proposition 1 du texte de Lê, c'est-à-dire:
comprendre les notions utilisées dans son énoncé et sa preuve
(point d'inflexion, multiplicité d'intersection entre une courbe et une droite,
triangle d'inflexion),
comprendre l'énoncé,
(essayer de) comprendre la preuve.
Consulter le libre de Fischer pour les notions de courbes.

En complément, vous pouvez méditer sur le théorème de Bezout,
la notion de courbe Hessienne, et le nombre de points d'inflexions
sur une quadrique.

\subsection{Semaine 4}
\label{sub:semaine_4}

Comprendre la Proposition 3, le  Corollaire 4 et le Théorème 5 du texte de Lê.



\subsection{Bibliographie}
\label{sub:bibliographie}

\begin{itemize}
	\item \href{../../ref_lecture/perrin_klein.pdf}{La quartique de Klein
		et le groupe simple d’ordre 168},
		Daniel Perrin.
	\item \href{../../ref_lecture/le.pdf}{La Quartique de Klein},
		François Lê.
		Un rapport de mémoire inspiré du texte de Perrin,
		peuvent se lire de manière complémentaire.
	\item \href{../../ref_lecture/fischer.pdf}{Plane Algebraic Curves},
		Gerd Fischer.
		Très bonne référence pour les courbes planes
		et le plan projectif.
	\item \href{../../ref_lecture/perrin_alg.pdf}{Cours d'algèbre},
		Daniel Perrin.
		Les chapitres I (groupes finis) et IV (groupes linéaires).
	\item \href{../../ref_lecture/levy.pdf}
		{The Eightfold way: the beauty of Klein's quartic curve},
		édité par Silvio Levy.
	\item \href{../../ref_lecture/bavard.pdf}{La surface de Klein},
		Christophe Bavard.
		La première partie présente très (trop) rapidement la surface de
		Klein et son groupe d'automorphisme.
		Dur d'y apprendre quelque chose, mais peut servir
		comme un résumé synthétique.
\end{itemize}
\section{La quartique de Klein}
\label{sec:quartique}

La quartique de Klein est la
\emph{courbe projective plane}
d'équation homogène:
\begin{equation}
	\label{eq:klein}
	x^3 y
	+
	y^3 z
	+
	z^3 x
	= 0
	.
\end{equation}
% On peut montrer que c'est une courbe
% \emph{lisse}.

La première étape est donc de comprendre ce qu'est une courbe projective plane.
Pour cela, il faut définir le \emph{plan projectif}
$ \PP^2(\C) $ 
(et plus généralement
la notion d'espace projectif).

Ensuite, il faut comprendre pourquoi une expression
du type \ref{eq:klein} définit une courbe.
Une telle expression s'appelle un \emph{équation polynomiale homogène}.

Comme nous le verrons, la quartique de Klein est intéressante parce
qu'elle est la plus symétrique possible, en un sens à préciser.
Pour comprendre cela, il faut introduire le groupe
$ \mathrm{PGL}(3,\C) $
des \emph{transformations projectives}
du plan.
Ensuite, il faudra étudier le sous-groupe
de ces transformations projectives qui préservent
la quartique de Klein.

Ce groupe est remarquable car c'est l'unique groupe
simple (non-abélien) d'ordre 168, et c'est le 2eme plus petit groupe
simple (non-abélien) après le groupe alterné $ A_{5} $ d'ordre 60.
De plus, c'est l'ordre maximale d'un groupe qui préserve une quartique.
Réciproquement, une quartique qui a un groupe de symétrie d'ordre 168
et forcément la quartique de Klein.

\section{Théorie des groupes}
\label{sec:theorie_des_groupes}

Exercices:

Bestiaire de groupes finis:
\begin{enumerate}
	\item Il existe un unique groupe simple d'ordre 60.
	\item Il existe un unique groupe simple d'ordre 168
		(dur, voir le Cours d'algèbre de Perrin page 115).
	\item Ce sont les seules groupes simples non-abélien d'ordre
		$ \le 168 $.

\end{enumerate}

Groupes linéaires:
\begin{enumerate}
	\item Calculer le cardinal de $ \mathrm{(P)SL}(n, K) $
		où $ K $ est un corps fini.
	\item $ \mathrm{PSL}(2, K) $ est simple si $ K $ a au moins 4 (?)
		élements \textbf{(dur)}.
\end{enumerate}

Groupes de symétries:
\begin{enumerate}
	\item Décrire le groupe des isométries du plan euclidien.
	\item Trouver le groupe d'isométries du plan qui
		préservent un polygone régulier à
		$ n $-côtés.
	\item Trouver le groupe d'isométries du plan qui
		préservent l'ensemble
		$ \Z \times \left\{ 0 \right\} \subset \R^2 $.
	\item Décrire le groupe des isométries de la sphère
		$ \mathbb{S}^{2} =
		\left\{
			x^{2}
			+
			y^{2}
			+
			z^{2}
			=1
		\right\}
		\subset \R^{3} 
		$.
	\item Déterminer les groupes d'isométries
		des solides de Platon.
	\item Déterminer les sous-groupes finis
		des isométries de
		$ \mathbb{S}^{2} $
		\textbf{(dur)}.
\end{enumerate}

\subsection{Le groupe $ \mathrm{PSL}(2, F_7) $ }
\label{sub:psl2f7}

Un dévissage de $ G =  \mathrm{PSL}(2, F_7) $.
Pour comprendre ce groupe, on détermine tout ses
sous-groupes et l'action de $ G $ sur ces sous-groupes.
Pour cela, on commence par l'étude de ses $ p $-Syllow,
pour $ p=2,3,7 $.

\begin{enumerate}
	\item Montrer que le cardinal de $ G $
		est $ 168 = 2^3 \cdot 3 \cdot 7 $.
	\item Montrer que $ G $ agit transitivement sur la droite projective
		$ \PP^{1}(F_{7}) $,
		c'est à dire l'ensemble des droites
		de $ F_{7}^{2} $.
	\item Montrer que $ G $ agit transitivement sur les
		paires de points distincts de
		$ \PP^{1}(F_{7}) $.
\end{enumerate}

On considère les sous-groupes de $ G $ suivant:
\begin{itemize}
	\item Le groupe $ P $ des matrices triangulaires supérieur.
	\item Le groupe $ A \subset P $ des matrices diagonales.
	\item Le groupe $ U \subset P $ des matrices triangulaires supérieur
		avec des $ 1 $ sur la diagonale.
\end{itemize}

\begin{enumerate}
	\item Calculer les ordres des sous-groupes
		$P, A $ et $ U $ (attention, on est dans
		$ \mathrm{PSL} $ et pas $ \mathrm{SL} $!).
	\item Montrer que $ P $ est le stabilisateur
		d'un point de
		$ \PP^{1}(F_{7}) $
		(c'est à dire d'une droite de $ F_{7}^{2} $.
	\item Montrer que $ A $ est le stabilisateur
		d'une paire de points distincts de
		$ \PP^{1}(F_{7}) $.
	\item Montrer que $ P $ est le normalisateur
		de $ U $.
	\item Calculer le normalisateur de $ A $.
	\item En déduire le nombre de $ 3 $-Syllow et de $ 7 $-Syllow.
\end{enumerate}

Formes quadratiques sur $ F_{7} $.
On définit $ q(x,y) = x^2+y^2 $,
qui est une forme quadratique
sur $ F_{7}^2 $.
Remarquer que
$ \mathrm{SL}(2, F_{7}) $
agit sur l'ensemble des formes quadratiques
sur $ F_{7}^2 $
par précomposition:
$ (M, q) \mapsto q \circ M^{-1} $.
\begin{enumerate}
	\item Déterminer le sous-groupe
		de $ \mathrm{SL}(2, F_{7}) $
		des matrices
		qui préservent $ q $.
	\item Déterminer le sous-groupe
		de $ \mathrm{SL}(2, F_{7}) $
		des matrices
		qui préservent
		$ \left\{ q, -q \right\} $.
\end{enumerate}

Nombres complexes sur
$ F_{7} $.

\begin{enumerate}
	\item Montrer que $ -1 $ n'est pas un carré
		dans $ F_{7} $.
	\item Montrer qu'il existe un plus petit corps
		qui contient $ F_{7} $ et dans le lequel
		$ -1 $ est un carré. Montrer que ce corps
		est de cardinal $ 7 \cdot 7 $.
		On l'appelera le corps de nombres complexes
		sur $ F_{7} $.
\end{enumerate}

\subsection{Le groupe des homographies $\mathrm{PGL}(3, \C)$}
\label{sub:homographies}

Le groupe des homographies, ou bien le groupe des transformations projectives,
du plan $ P^{2} (\C) $
est le groupe
quotient
$ \mathrm{PGL}(3, \C)
=
\mathrm{GL}(3, \C)
/
\left\{ \pm 1 \right\}
$.
On étudie ici quelques propriétés et sous-groupe intéressants de ce groupe.
On définit les points du plan projectif
$ a = [1, 0, 0] $,
$ b = [0, 1, 0] $,
et
$ c = [0, 0, 1] $

\begin{enumerate}
	\item Expliciter le stabilisateur du point
		$ a $
		dans
		$ \mathrm{PGL}(3, \C) $.
	\item En déduire le sous-groupe de
		$ \mathrm{PGL}(3, \C) $
		qui fixe $a $, $ b $ et $ c $.
	% \item Trouver le sous-groupe de
	% 	$ \mathrm{PGL}(3, \C) $
	% 	qui fixe $ a $ et qui laisse stable
	% 	l'ensemble $ \left\{ b,c \right\} $.
\end{enumerate}

On note maintenant $ K $
la quartique de Klein d'équation
\begin{equation}
	\label{eq:klein}
	x^3 y
	+
	y^3 z
	+
	z^3 x
	= 0
	.
\end{equation}
On note $ G $
le sous-groupe de
$ \mathrm{PGL}(3, \C) $
qui fixe $ K $.

\begin{enumerate}
	\item Trouver le stabilisateur de $ a $ dans $ G $.
\end{enumerate}


\section{Courbes planes}
\label{sec:courbes_planes}

\subsection{Coniques}
\label{sub:coniques}

Une \emph{conique} est une courbe projective plane
définie par un polynome homogène de degré 2.

Montrer qu'à transformation projective près, il n'existe que
3 coniques distinctes: 
\begin{itemize}
	\item La double droite $ x^2 = 0 $
	\item La paire de droite $ xy = 0 $
	\item Le cercle $ x^2+y^2+z^2=0 $
		(ce cas est dit \emph{non-dégénéré}).
\end{itemize}
On pourra utiliser la classification des formes quadratiques sur
$ \C $.
Comparer avec la situation réelle.

Montrer qu'une conique non-dégénéré est \emph{rationnelle},
c'est à dire qu'il existe une bijection
de $ \PP^1 (\C) $ vers cette conique.
\emph{Indication}: dans une carte affine où la conique est d'équation
$ x^2+y^2=1 $, regarder les droites qui passent par $ (0,1) $.
Chacune de ces droites recoupe la conique en un unique autre point 
(à part la droite verticale).

Déterminer le groupe des symétries d'une conique non-dégénéré.

\end{document}
