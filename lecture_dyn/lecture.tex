\documentclass{article}
\usepackage{mypackage}
\begin{document}

\section{Notions dynamiques}
\label{sec:notions_dynamiques}

On peut définir de manière très générales les notions
dites "dynamiques" suivantes.
Soit $ X $ un ensemble
et $ f : X \to X $
une application de $ X $ dans lui même.
On dit parfois que
$ (X,f) $
est un système dynamique.

\begin{itemize}
        \item
      	  Pour un entier $ n \in \mathbb{N} $,
      	  \emph{l'itéré} $n$-ième de $ f $,
      	  noté $ f^{n} $,
      	  est la composition
      	  $ f \circ \cdots \circ f $
      	  (avec $ n $ termes).

        \item Un \emph{point fixe}
      	  de $ f $ 
      	  est un point $ x \in X $
      	  tel que $ f(x) = x $.
        \item Un
      	  \emph{point périodique}
      	  de $ f $
      	  est un point $ x \in X $
      	  tel qu'il existe $ n $
      	  tel que $ f^n(x) = x $.
      	  Autrement dit,
      	  c'est le point fixe
      	  d'une certaine itéré
      	  de $ f $.
        \item Un sous-ensemble
      	  $ Y \subset X $
      	  est \emph{stable}
      	  (ou invariant)
      	  si $ f(Y) \subset Y $.
\end{itemize}

Soit
$ (Y,g) $
un autre système dynamique.
On dit que
$ f $ est
\emph{conjuguée}
à
$ g $
si il existe
une bijection
$ \varphi : X \to Y $
telle que
$ \varphi \circ f = g \circ \varphi $.
Intuitivement,
$ g $ est obtenu à partir de $ f $
en faisant le "changement de variable"
$ y = \varphi(x) $.
Remarquer que c'est une relation d'équivalence.

\section{Exercices}
\label{sec:exercices}

On trouvera ici quelques exercices
"ouverts", c'est à dire sans réponse précise attendue,
pour s'entrainer à manipuler les notions dynamiques.

\textbf{Dynamique dans un ensemble fini.}
Soit $ X $ un ensemble fini
et $ f : X \to X $.
Le \emph{graphe dynamique}
de $ f $ est le graphe
dont les sommets sont les points de $ X $
et où l'on relie deux points
$ x $ et $ y $ par une arrête si
$ f(x) = y $.
Décrire les différentes allures
possibles pour un graphe dynamique
sur un ensemble fini.

\textbf{Polynômes de degré 1}.
Soit $ f $ un polynôme complexe de degré 1,
c'est à dire que
$ f(x) = az + b $
avec $ a $ et $ b $
des complexes tels que
$ a \neq 0 $.
Étudier la dynamique
de
$ f : \mathbb{C} \to \mathbb{C} $
(par exemple: ses points fixes, périodiques...).
Comment cette dynamique varie avec le paramètre
$ a $ ?
Trouver une conjugaison
(c'est à dire un changement de variable)
qui réduise $ f $ à une forme la plus simple
possible.

\textbf{Degré 2.}
Soit $ f(z) = z^{2} $.
Étudier la dynamique
de
$ f : \mathbb{C} \to \mathbb{C} $.

% \section{Résultats de dynamique complexe}
% \label{sec:resultats_de_dynamique_complexe}



\end{document}
